\documentclass[12pt, letterpaper]{article}
\usepackage[utf8]{inputenc}

\title{Octopoints}
\author{Tommaso Pavani}
\date{Gennaio 2022}

\begin{document}
    \maketitle
    \section{Introduzione}
    \subsection{Obiettivo}
    L'applicazione mobile dovrà tenere il conteggio dei punti di una partita.
    Dovrà essere il più possibile generica in modo da poter essere usata per qualsiasi gioco da tavolo, in squadre o 1vs1.
    \subsection{Features}
    L'app avrà le seguenti funzionalità principali:
    \begin{itemize}
        \item Tenere il conteggio dei punti per ogni giocatore nella partita
        \item Possibilità di avere più partite in corso
    \end{itemize}
    E qualche funzionalità extra:
    \begin{itemize}
        \item Possibilità di gestire partite in squadre
        \item Alcune statistiche sui giocatori
    \end{itemize}
    \subsection{Approccio}
    \emph{Per semplicità di sviluppo, l'app avrà un database locale.}
    \newline
    Ogni partita ovviamente dovrà avere, oltre ad un codice identificativo e un nome, delle regole:
    \begin{itemize}
        \item Punteggio da raggiungere (la partita termina quando uno o più giocatori raggiungono il punteggio fissato)
        \item Numero dei vincitori (in alcuni giochi ci possono essere n vincitori)
        \item Modalità di gioco (per generalizzare il più possibile, definiamo due macro modalita: un giocatore che arriva al punteggio fissato può vincere o perdere)
    \end{itemize}
    Per quanto riguarda la gestione di partite in squadre, ci limitiamo a gestire ogni partita come se fosse in squadre. Nel caso non sia un gioco di squadra, questa sarà composta da un solo giocatore.
    Questo approccio ci permette di gestire anche partite particolari (es. 1vs2).
    I punteggi della partita saranno quindi salvati per la squadra, in quanto uguali per tutti i suoi componenti.
    
    Le statistiche sui giocatori verranno aggiornate al termine di ogni partita, alla quale partecipano, mantenendone l'esito (vinta persa o pareggiata).
\end{document}